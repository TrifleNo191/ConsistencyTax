\documentclass[11pt]{article}

\usepackage[a4paper,margin=1in]{geometry}
\usepackage{amsmath,amssymb,amsfonts,amsthm}
\usepackage{bm}
\usepackage{siunitx}
\usepackage{booktabs}
\usepackage{graphicx}
\usepackage{tikz}
\usepackage{pgfplots}
\pgfplotsset{compat=1.18}
\usepackage{microtype}
\usepackage{hyperref}
\usepackage{cleveref}
\usepackage{csquotes}
\usepackage[backend=biber,style=alphabetic,maxcitenames=3,maxbibnames=10]{biblatex}

\addbibresource{references.bib}

\newcommand{\KL}{D_{\mathrm{KL}}}
\newcommand{\E}{\mathbb{E}}
\newcommand{\ct}{\mathrm{CT}}
\newcommand{\kT}{k_B T \ln 2}

\title{Supplement to ``Consistency Tax and Epistemic Thermodynamics''}
\author{Andra\v{z} \DJ uri\v{c}}
\date{\today}

\begin{document}
\maketitle

\section{Coherent Reference Construction: Examples}

\subsection{Binary channel with mismatched beliefs}

We illustrate $P^\*$ for a binary symmetric channel with crossover probability $p$ and a decision mechanism based on subjective model $q$.

\begin{enumerate}
\item Environment: bits $X\in\{0,1\}$, noise $P(Y\neq X)=p$.
\item System: maintains belief $Q(X)$ and decision rule $\hat X(Y)$ optimized for $q\neq p$.
\item Reference: uses correct $p$, same architecture, same loss.
\end{enumerate}

For fixed architecture and average power budget, $P^\*$ corresponds to choosing $Q=P$ and decision rule minimizing expected loss.
CT reduces to $\kT\KL(P\Vert Q)$ plus erasure contributions from corrections.

\subsection{Linear-Gaussian control}

Consider a linear plant with quadratic cost and a controller using estimate $\hat A$ of true dynamics $A$.
Following standard LQG results and nonequilibrium work bounds, we obtain an excess cost proportional to a KL divergence between path distributions under $(A,B)$ vs $(\hat A,B)$.
This implements CT$_1$.

\section{Three-Term Inequality Details}

\subsection{CT$_1$: Model--world}

Using the Kullback inequality for driven Markov processes, the minimal additional dissipation for using mismatched protocol $Q$ in environment $P$ over horizon $T$ obeys
\[
W_{\text{excess}} \ge k_B T \KL(P \Vert Q),
\]
for appropriate path measures.
We restate this as $\ct_1 \ge \kT\,\lambda_1 \KL(P\Vert Q)$ for event rate $\lambda_1$.
Full derivations follow \cite{Jarzynski1997,Crooks1999,Seifert2012}.

\subsection{CT$_2$: Model--signal}

We sketch finite-state constructions:
assume an internal register storing $M$ and an outward channel required to emit $R$ that differs from $M$.
If external constraints forbid revealing $M$, the system must either:
(i) compute $R$ from $M$ via irreversible mapping, or
(ii) maintain parallel consistent and outward-facing states.
Both induce erasures when $M$ changes or contradictions are patched.

For simple binary examples one can show:
\[
\ct_2 \ge \kT\,\lambda_2\,\KL(Q\Vert R)
\]
up to constants.
A general proof for arbitrary channels is left as an open problem.

\subsection{CT$_3$: Erasure}

Erasure accounting follows directly from Landauer.
We emphasize that only erasures attributable to maintaining misalignment are counted toward CT.

\section{Toy POMDP Simulation}

We formalize the bias-vs-CT experiment used illustratively in the main text.

\subsection{Environment}

Two hidden states $W\in\{0,1\}$ with prior $\Pr(W=1)=0.5$.
At each time, observation $O\in\{0,1\}$ with
$\Pr(O=W)=0.9$.

\subsection{Agent}

Maintains biased belief
\[
\Pr_\theta(W=1\mid O) \propto
\exp(\theta)\Pr(O\mid W=1)
\]
with bias parameter $\theta$.
For $\theta=0$ the belief is Bayes-optimal.

\subsection{Consistency Tax}

For each $\theta$ we compute
\[
\ct(\theta) = \KL\big(P^\*(W\mid O)\,\Vert\,P_\theta(W\mid O)\big),
\]
averaged over $O$.
The resulting curve is symmetric in $\theta$ with minimum at $0$; large biases incur higher CT.

\subsection{Pseudocode}

\begin{verbatim}
for theta in bias_values:
    ct = 0.0
    for w in {0,1}:
        for o in {0,1}:
            p_star = P(w) * P(o|w)
            p_post_star = bayes_posterior(w|o)
            p_post_theta = biased_posterior(w|o, theta)
            ct += p_star * log(p_post_star / p_post_theta)
    results[theta] = ct
\end{verbatim}

\section{LLM CTI Benchmark: Specification}

\subsection{Design}

\begin{itemize}
\item Model: fixed architecture and weights.
\item Prompts: curated factual, mathematical, and logical questions with known answers.
\item Conditions:
truthful; constrained-false; style-matched control.
\item Metrics: tokens, wall-time, GPU energy, average NLL.
\end{itemize}

\subsection{Analysis}

Estimate $\mathrm{CT}_Q = C_{\text{false}}-C_{\text{truth}}$ for each task family.
Report mean differences with confidence intervals.
Include ablations for sampling temperature and prompt length.

\section{Agent-Based Evolutionary Simulation}

\subsection{Outline}

Agents have parameters:
accuracy (world-model quality),
honesty (alignment of $R$ with $Q$),
compartmentalization (ability to localize contradictions).
Per-round payoff = task reward $-$ estimated CT.

\subsection{Pseudocode}

\begin{verbatim}
for gen in generations:
    for agent in population:
        P = sample_environment()
        Q = agent.model(P)
        R = agent.signal(Q)
        ct = estimate_CT(P, Q, R, params)
        reward = task_payoff(P, R) - ct
    population = reproduce(population, reward)
\end{verbatim}

Explore regimes where low-CT agents dominate vs where deceptive strategies persist.

\section{Organizational CT Index}

\subsection{Definition}

Combine:
(1) discrepancy between internal metrics and audited outcomes;
(2) discrepancy between internal docs and external reports;
(3) churn and incident rates.

Normalize into index in $[0,1]$.
Hypothesis: higher CT index predicts higher costs and risk.

\section{Notes on Evidence Grading}

Tables in the main text summarise which components rest on proofs (e.g.\ Landauer, fluctuation theorems) and which remain conjectural (CT$_2$ generality, cross-scale aggregation).

\printbibliography

\end{document}
