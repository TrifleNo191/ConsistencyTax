\documentclass[11pt]{article}

\usepackage[a4paper,margin=1in]{geometry}
\usepackage{amsmath,amssymb,amsfonts,amsthm}
\usepackage{bm}
\usepackage{siunitx}
\usepackage{microtype}
\usepackage{hyperref}
\usepackage{cleveref}
\usepackage{csquotes}
\usepackage[backend=biber,style=alphabetic,maxcitenames=3,maxbibnames=10]{biblatex}

\addbibresource{references.bib}

\newcommand{\KL}{D_{\mathrm{KL}}}
\newcommand{\E}{\mathbb{E}}
\newcommand{\ct}{\mathrm{CT}}
\newcommand{\kT}{k_B T \ln 2}

\theoremstyle{plain}
\newtheorem{theorem}{Theorem}
\newtheorem{proposition}{Proposition}
\theoremstyle{definition}
\newtheorem{definition}{Definition}
\newtheorem{assumption}{Assumption}
\theoremstyle{remark}
\newtheorem{remark}{Remark}

\title{The Consistency Tax:\\
A Thermodynamic Bound on Misalignment}

\author{Andra\v{z} \DJ uri\v{c}\thanks{Independent researcher. This manuscript is released as a precise working hypothesis and may serve as the author's final public synthesis of the Consistency Tax program for the near future.}}

\date{\today}

\begin{document}
\maketitle

\begin{abstract}
Physical information-processing systems---biological, artificial, institutional---operate under thermodynamic constraints while maintaining internal models, emitting signals, and pursuing objectives.
When these elements are systematically misaligned, intuition suggests additional energetic and computational overhead.
This paper states the Consistency Tax (CT) as a single, testable object.

Let $W$ denote world states, $M$ internal models, $S$ signals or actions, and $G$ objectives.
Let $P(W,M,S,G)$ be the realized joint distribution induced by a system, and let $P^\*(W,M,S,G)$ be a coherent reference joint that, under the same environment and architectural constraints, minimizes expected task loss and resource cost subject to mutual consistency between $W,M,S,G$.
We define
\[
\ct = \kT\,\KL\!\big(P(W,M,S,G)\,\Vert\,P^\*(W,M,S,G)\big),
\]
interpreted as a Landauer-scaled lower bound on the avoidable energetic overhead due to misalignment.

This formulation adds no new physics: it is an organizing principle built from KL divergence, Landauer's bound, and standard stochastic thermodynamics.
Its novelty lies in (i) the joint treatment of world, model, signals, and goals; (ii) the use of a constrained coherent reference as baseline; and (iii) a mechanistic three-term inequality that connects CT to model--world mismatch, model--signal distortion, and misalignment-attributable erasure.
We outline concrete predictions for large language models, human deception, microbial chemotaxis, and organizations, and specify conditions under which CT should be judged trivial or false.
\end{abstract}

\section{Canonical Object}

\subsection{Realized and reference joints}

\begin{definition}[Realized joint]
For a system operating in a stationary environment on a timescale where empirical frequencies are meaningful, let
\[
P(W,M,S,G)
\]
denote the joint distribution over:
world states $W$, internal model states $M$, signals or actions $S$, and objectives or constraints $G$,
induced by the actual physical implementation under fixed external conditions.
\end{definition}

\begin{definition}[Coherent reference joint]
Let $\mathcal{C}$ be the set of feasible joint distributions $\tilde P(W,M,S,G)$ such that:
\begin{enumerate}
\item[(i)] they face the same environment and external interface as $P$;
\item[(ii)] they respect the same physical and architectural constraints (up to reversible refinements);
\item[(iii)] they maintain mutual consistency between $W,M,S,G$ (no deliberate contradictions in steady state);
\item[(iv)] they minimize a specified functional
\[
\mathcal{J}(\tilde P) = \E_{\tilde P}[L(W,S,G)] + \alpha E_{\text{thermo}}(\tilde P),
\]
where $L$ is task loss and $E_{\text{thermo}}$ a resource cost (e.g.\ energy, compute time).
\end{enumerate}
Any minimizer $P^\*(W,M,S,G)\in\mathcal{C}$ is called a \emph{coherent reference joint}.
\end{definition}

This defines $P^\*$ as the best available coherent baseline for the same job and hardware class.
It is architecture- and task-relative, not idealized omniscience.

\begin{definition}[Consistency Tax]
The \emph{Consistency Tax} of a realized system is
\begin{equation}
\ct \;:=\; \kT\,\KL\!\big(P(W,M,S,G)\,\Vert\,P^\*(W,M,S,G)\big).
\label{eq:ct_def}
\end{equation}
\end{definition}

\begin{proposition}[Zero-tax coherence]
$\ct = 0$ if and only if $P(W,M,S,G)=P^\*(W,M,S,G)$ almost everywhere.
\end{proposition}

\noindent
\textbf{Claim CT-1.}
Equation~\eqref{eq:ct_def} is the canonical definition of CT.
It measures an avoidable energetic overhead relative to a coherent reference under shared constraints.  
\emph{Confidence:} 0.9 (\emph{definition}); \emph{Evidence:} A/B.

\section{Operational Decomposition}

Many systems are naturally described by:
environment statistics $P$,
internal models $Q$,
and outward behaviour $R$.
We show how \eqref{eq:ct_def} yields a concrete lower bound in terms of these components.

\begin{assumption}[P--Q--R setting]
We assume:
(i) an environment with distribution $P$ over task-relevant events;
(ii) an internal model $Q$ used for prediction or control;
(iii) a channel $R$ from internal states to signals/actions;
(iv) implementation by finite-state devices obeying Landauer's principle \cite{Landauer1961,Bennett1982}.
\end{assumption}

\begin{theorem}[Three-term CT inequality (template)]
\label{thm:three}
Under the P--Q--R setting, suppose that:
\begin{enumerate}
\item[(a)] the coherent reference $P^\*$ corresponds to $Q=P$, $R$ aligned with $Q$ up to task-optimal stochasticity, and no misalignment-specific erasures;
\item[(b)] predictive and control events driven by $P$ occur at rate $\lambda_1(t)$,
messages or decisions based on $Q$ at rate $\lambda_2(t)$;
\item[(c)] $r_{\mathrm{erase}}(t)$ bits/s of logically irreversible erasures are attributable to maintaining discrepancies between $P$, $Q$, and $R$.
\end{enumerate}
Then, for time windows where these rates are approximately constant,
\begin{equation}
\frac{\ct(t)}{\Delta t}
\;\ge\;
\kT
\Big[\lambda_1(t)\,\KL(P\Vert Q)
+ \lambda_2(t)\,\KL(Q\Vert R)
+ r_{\mathrm{erase}}(t)\Big].
\label{eq:three_term}
\end{equation}
\end{theorem}

\begin{proof}[Proof sketch]
For the model--world term, mismatched protocols incur excess work bounded below by $k_B T$ times a KL divergence between path measures \cite{Jarzynski1997,Crooks1999,Seifert2012}; coarse-graining to event rate $\lambda_1$ yields the first term.
For erasures, Landauer's principle implies a minimal dissipation $n\kT$ for $n$ bits erased; restricting to erasures attributable to resolving or hiding inconsistencies yields the third term.
The model--signal term reflects the cost of implementing a stochastic map that systematically distorts $Q$ into $R$ in irreducible ways.
Finite-state constructions show this cost scales at least with $\KL(Q\Vert R)$ up to constants in simple settings; a fully general bound remains conjectural and is marked as such.
Formally, \eqref{eq:three_term} is a conservative inequality consistent with \eqref{eq:ct_def}.
\end{proof}

\noindent
\textbf{Claim CT-2.}
Equation~\eqref{eq:three_term} is a valid template:
CT$_1$ (model--world) and CT$_3$ (erasure) are grounded in established results (Confidence 0.85, Evidence B/A);
CT$_2$ (model--signal) is conjectural but supported by toy constructions and cognitive data (0.7, C).
Any counterexample consistent with the assumptions should refine or restrict this bound.

\paragraph{Rarity bound.}
If the coherent reference assigns probability $p$ to a constrained signal set $S$, any mechanism that forces outcomes into $S$ incurs at least $\log_2(1/p)$ bits of additional specification, hence
\[
\ct_{\mathrm{rarity}} \,\ge\, \kT \log_2\!\frac{1}{p}.
\]
(Confidence 0.95, Evidence A; coding-theoretic.)

\section{Delta vs Prior Work}

CT is intentionally modest in ontology and sharp in structure.

\subsection*{Free Energy Principle and predictive processing}

FEP-type approaches describe systems as minimizing a variational bound on surprise \cite{Friston2010FEP}.
CT differs by:
(i) explicitly including $S$ and $G$ and asking about misalignment across all four;
(ii) defining a reference $P^\*$ that enforces coherence under the same constraints;
(iii) focusing on \emph{extra} dissipation relative to that baseline.
FEP can be seen as one ingredient in constructing $P^\*$; CT is orthogonal bookkeeping.

\subsection*{Thermodynamics of computation}

Landauer, Bennett, and successors already bound the cost of erasure and model mismatch \cite{Landauer1961,Bennett1982,Jarzynski1997,Crooks1999,Seifert2012}.
CT places those results in a semantic wrapper:
instead of arbitrary bits, we track bits that carry disagreement between $W,M,S,G$.
The delta is not new physics, but a principled way to say:
``\emph{this} part of your energy bill is the price of inconsistency.''

\subsection*{Economics, deception, and organizational theory}

Work on deception and transaction costs documents that lying, misreporting, and misaligned incentives tend to be costly.
CT provides a physical upper/lower bounding language and a program to separate strategic benefits from thermodynamic overhead.
If empirical work shows stable, cheap misalignment under full accounting, CT must be narrowed.

\section{Predictions and Tests}

The CT program stands or falls on whether it adds non-trivial predictive power.
We state three core predictions; each is falsifiable.

\subsection{Prediction 1: Constrained-false outputs cost more in AI systems}

\textbf{Claim.}
For a fixed model, hardware, and decoding regime, generating constrained-false outputs that preserve internal-external inconsistency has higher expected energy or latency than truthful baselines with matched length and difficulty.
(Confidence 0.7, Evidence C.)

\textbf{Protocol (sketch).}
Select a large set of factual and logical prompts.
For each:
(i) generate truthful answers;
(ii) force the model (via constraints/prompts) into inconsistent narratives.
Measure tokens, wall-clock, and energy (via standard counters).
Control for output length and randomness.
A robust null or reverse effect under full accounting would directly challenge CT$_2$ for this class.

\subsection{Prediction 2: Sustained human deception is measurably dearer}

\textbf{Claim.}
In carefully matched laboratory tasks, sustained deception requires greater cognitive/energetic effort than truthful reporting.
(0.8, B; consistent with existing literature.)

\textbf{Test.}
Within-subject design; record RT, accuracy, and physiological proxies.
If deception is systematically no more costly when controlling for difficulty, CT$_2$ is weakened for human cognition.

\subsection{Prediction 3: Conflicting cues waste energy in controlled biological systems}

\textbf{Claim.}
Microbial populations exposed to systematically conflicting chemo-signals incur higher ATP consumption per unit biomass than those receiving coherent signals.
(0.6, C.)

\textbf{Test.}
Microfluidics with controlled gradients; measure growth and energetics.
A clear absence of overhead despite persistent misalignment would constrain CT$_1$ in that setting.

\section{Interpretation and Limits}

\subsection{What CT is}

CT is a \emph{conditional} thermodynamic bound:
a way to attribute part of a system's dissipation to misalignment between its world, models, signals, and goals, relative to a coherent feasible alternative.

\subsection{What CT is not}

CT is not:
new physics; a universal law that ``truth always wins''; or a moral theorem.
Game-theoretic analyses readily exhibit regimes where the benefits of deception exceed CT, especially with weak feedback or externalization of costs.

\subsection{Non-triviality requirement}

\textbf{Claim CT-3.}
If, in the relevant domains, CT:
(i) reduces to existing complexity or loss metrics; or
(ii) fails to predict any measurable overhead where formal misalignment is present; or
(iii) is routinely violated under honest thermodynamic accounting,
then CT should be considered trivial or false and discarded or sharply scoped.
(Confidence 0.95, Evidence A as a methodological stance.)

\section{Author's Note}

This paper isolates the densest statement of the Consistency Tax program I can currently defend.

The construction is intentionally minimal:
a single KL-based definition relative to a coherent reference joint under shared constraints; a mechanistic but partially conjectural three-term inequality; and a short list of experiments that can confirm, refine, or reject the framework.
No surrounding narrative is required.

From this point, further progress is primarily a matter for proofs and data:
formalizing the model--signal term,
quantifying CT in real machines and organisms,
and demonstrating either clear predictive value or systematic failure.
If CT survives those tests, it may join the standard toolkit for reasoning about information, energy, and alignment.
If it does not, this document should make it easy to see exactly what was meant, and why it failed.

\printbibliography

\end{document}
