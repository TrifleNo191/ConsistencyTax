\documentclass[11pt]{article}

\usepackage[a4paper,margin=1in]{geometry}
\usepackage{amsmath,amssymb,amsfonts,amsthm}
\usepackage{bm}
\usepackage{siunitx}
\usepackage{booktabs}
\usepackage{graphicx}
\usepackage{tikz}
\usepackage{pgfplots}
\pgfplotsset{compat=1.18}
\usepackage{microtype}
\usepackage{hyperref}
\usepackage{cleveref}
\usepackage{csquotes}
\usepackage[backend=biber,style=alphabetic,maxcitenames=3,maxbibnames=10]{biblatex}

\addbibresource{references.bib}

\newcommand{\KL}{D_{\mathrm{KL}}}
\newcommand{\E}{\mathbb{E}}
\newcommand{\ct}{\mathrm{CT}}
\newcommand{\kT}{k_B T \ln 2}

\theoremstyle{plain}
\newtheorem{theorem}{Theorem}
\newtheorem{proposition}{Proposition}
\newtheorem{corollary}{Corollary}
\theoremstyle{definition}
\newtheorem{definition}{Definition}
\newtheorem{assumption}{Assumption}
\newtheorem{remark}{Remark}

\title{Consistency Tax and Epistemic Thermodynamics:\\
The Energetic Cost of Misalignment}

\author{Andra\v{z} \DJ uri\v{c}\thanks{Independent researcher. This framework is released as a working hypothesis, with explicit tests and failure modes, for community evaluation and extension.}}

\date{\today}

\begin{document}
\maketitle

\begin{abstract}
Physical information-processing systems---from cells and brains to artificial models and institutions---operate under thermodynamic constraints while maintaining internal models, emitting signals, and pursuing objectives.
When these elements are systematically misaligned, intuition suggests additional energetic and computational overhead.
This paper consolidates the \emph{Consistency Tax} (CT) program into a single formalism and experimental roadmap.

Let $W$ denote world states, $M$ internal models, $S$ signals/actions, and $G$ objectives.
Let $P(W,M,S,G)$ be the realized joint distribution induced by a system, and let $P^\*(W,M,S,G)$ be a coherent reference joint that, under the same environment and constraints, minimizes expected task loss and resource cost subject to mutual consistency between $W,M,S,G$.
We define the Consistency Tax as
\[
\ct = \kT\,\KL\!\big(P(W,M,S,G)\,\Vert\,P^\*(W,M,S,G)\big),
\]
interpreted as a Landauer-scaled lower bound on the avoidable energetic overhead due to misalignment.

For systems characterized by environment statistics $P$, internal models $Q$, and outward behaviour $R$, we derive an operational three-term inequality relating CT to (i) model--world mismatch, (ii) model--signal distortion, and (iii) logically irreversible erasure rates.
We position CT relative to established results in thermodynamics of computation, stochastic thermodynamics, the Free Energy Principle, and transaction-cost economics, highlighting both overlaps and deltas.
We then specify falsifiable predictions and concrete measurement protocols for large language models, human deception tasks, microbial chemotaxis, organizational coherence, and agent-based evolutionary simulations.

CT is presented explicitly as a testable template rather than a new physical law.
If CT fails to explain variance beyond existing complexity and loss measures, or if stable low-cost misaligned systems are demonstrated under full accounting, the framework should be revised or discarded.
\end{abstract}

\section{Introduction}

Modern systems that sense, predict, and act---biological, artificial, and institutional---continually transform free energy into work, heat, and information.
They maintain internal models of their environment, communicate outwardly, and pursue stated or implicit objectives.
When these elements disagree, systems incur costs:
errors, rework, monitoring overhead, suppressed internal signals, and the need to erase and rewrite state.

The \emph{Consistency Tax} (CT) program asks:

\begin{enumerate}[label=(Q\arabic*),leftmargin=2em]
\item Can we quantify the minimal thermodynamic cost of sustained misalignment between world, models, signals, and goals?
\item Under what conditions is this cost large enough to influence fitness, performance, or stability?
\item How can such a quantity guide the design and diagnosis of AI systems, scientific processes, and institutions?
\end{enumerate}

The proposal builds only on established information theory and thermodynamics.
No new primitive law is introduced.
The central object is a Kullback--Leibler divergence between the realized behaviour of a system and a coherent baseline, scaled by the Landauer factor $\kT$.

This paper is designed as a self-contained, reproducible handoff of the CT framework.
All major claims are annotated with (i) a confidence level in $[0,1]$ and (ii) an evidence grade:
A (proof or multiple independent empirical sources),
B (single strong source or robust derivation with minor gaps),
C (suggestive or toy evidence),
D (speculative).

\subsection{Contributions}

\begin{enumerate}[leftmargin=2em]
\item \textbf{Unified definition (CT-1).}
We formalize CT as $\ct = \kT \KL(P\Vert P^\*)$ over $(W,M,S,G)$, with $P^\*$ a coherent, cost-minimizing reference joint.
(\emph{Confidence} 0.9, \emph{Evidence} A/B.)
\item \textbf{Operational three-term inequality (CT-4).}
For systems with environment $P$, internal models $Q$, and signals $R$, we derive a mechanistic lower bound decomposing CT into model--world, model--signal, and erasure terms.
(0.8, B.)
\item \textbf{Cross-domain template.}
We show how CT instantiates for AI models, neural and microbial systems, and organizations using the same formal object.
(0.6, C.)
\item \textbf{Falsifiable predictions and protocols.}
We provide at least five concrete experimental designs with explicit success/failure criteria, enabling independent tests.
(0.8, B.)
\item \textbf{Applications and tools.}
We outline CT-based diagnostics and regularizers for AI alignment, system design, and organizational auditing, and a CT-informed pipeline for scientific problem formulation.
(0.7, C.)
\end{enumerate}

If subsequent work shows CT offers no non-trivial predictive or explanatory gain beyond existing metrics, it should be narrowed or abandoned.
This rejection criterion is part of the proposal, not an afterthought.

\section{Background and Related Work}

\subsection{Thermodynamics of computation}

Landauer's principle establishes that logically irreversible operations dissipate at least $\kT$ of heat per erased bit \cite{Landauer1961}.
Bennett's work on reversible computation shows that, in principle, computation can be performed with arbitrarily low dissipation when logically reversible \cite{Bennett1982}.
Fluctuation theorems and nonequilibrium work relations further link KL divergences of path ensembles to minimal dissipation \cite{Jarzynski1997,Crooks1999,Seifert2012}.
CT relies directly on these results.

\subsection{Free Energy Principle and related frameworks}

The Free Energy Principle (FEP) and predictive processing treat adaptive systems as minimizing variational free energy or prediction error \cite{Friston2010FEP}.
Rational inattention, information bottleneck methods, and MDL trade accuracy against information-processing cost.
CT differs by:

\begin{itemize}[leftmargin=1.5em]
\item explicitly including signals/actions and objectives ($S,G$) alongside world and model;
\item isolating \emph{avoidable} overhead relative to a coherent reference under identical constraints;
\item emphasizing misalignment-induced erasure and bookkeeping costs.
\end{itemize}

\subsection{Deception, inconsistency, and organizational costs}

Empirical work indicates that lying and sustained deception often impose cognitive and behavioural costs \cite{DePaulo1996,Vrij2008}.
Organizational research documents losses from misreporting and misaligned incentives (To-verify).
CT does not claim these literatures prove its quantitative form; instead, they provide qualitative consistency checks.

\subsection{Scope}

CT is intended as a cross-domain template:
a way to quantify the physical cost of inconsistency that can be instantiated differently for AI systems, biological organisms, and institutions.
It is \emph{not} a claim that all aspects of truthfulness or ethics reduce to energy.

\section{Formalism of Consistency Tax}

\subsection{Canonical definition}

\begin{definition}[Realized joint]
For a system operating over a time scale long enough to define empirical distributions, let
$P(W,M,S,G)$ denote the joint distribution over:
world states $W$, internal model states $M$, signals/actions $S$, and objectives/constraints $G$,
induced by the actual physical implementation under fixed external conditions.
\end{definition}

\begin{definition}[Coherent reference joint]
Let $\mathcal{C}$ be the set of feasible joint distributions over $(W,M,S,G)$ that:
(i) are compatible with the same environment and external interface as $P$;
(ii) respect physical constraints of the architecture class;
(iii) maintain mutual consistency between $W,M,S,G$ (no deliberate contradictions);
(iv) minimize expected task loss plus a specified resource cost functional
(e.g., energy, compute time, hardware wear).

Any element $P^\*\in\mathcal{C}$ attaining this minimum is called a \emph{coherent reference joint}.
\end{definition}

We do not require uniqueness; if multiple $P^\*$ exist, CT uses any minimizer, and $\ct=0$ only when $P$ matches at least one such reference.

\begin{definition}[Consistency Tax]
The Consistency Tax is
\begin{equation}
\ct
\;:=\;
\kT\,
\KL\!\big(P(W,M,S,G)\,\Vert\,P^\*(W,M,S,G)\big).
\label{eq:ct}
\end{equation}
\end{definition}

\begin{proposition}[Zero-tax coherence]
$\ct = 0$ if and only if $P(W,M,S,G)=P^\*(W,M,S,G)$ almost everywhere.
\end{proposition}

\begin{proof}
Immediate from non-negativity and definiteness of $\KL(\cdot\Vert\cdot)$.
\end{proof}

\begin{remark}
CT is defined relative to a chosen architecture class, cost functional, and environment.
It is therefore \emph{conditional} and must always be reported with those conditions.
\end{remark}

\subsection{Operational P--Q--R decomposition}

In many systems it is useful to distinguish:
$P$: environment statistics over task-relevant variables,
$Q$: internal model or belief distribution,
$R$: distribution over signals/actions conditioned on internal state.

Under assumptions of ergodicity and fixed coding schemes,
we can express parts of the joint divergence in \cref{eq:ct} through divergences among $(P,Q,R)$.

\begin{theorem}[Three-term CT inequality (template)]
\label{thm:three_term}
Consider a system with:
\begin{itemize}[leftmargin=1.5em]
\item environment distribution $P$ for queries/events at rate $\lambda_1(t)$,
\item internal model $Q$ used to predict or control those events,
\item signals/actions $R$ generated from internal states at rate $\lambda_2(t)$,
\item misalignment-attributable irreversible erasures at rate $r_{\mathrm{erase}}(t)$.
\end{itemize}
Suppose that:
(i) environment $P$ is stationary on the measurement timescale;
(ii) control and communication are implemented via finite-state devices obeying Landauer's bound;
(iii) the coherent reference joint uses $Q=P$ and $R$ aligned with $Q$ up to task-optimal stochasticity.

Then the Consistency Tax rate satisfies
\begin{equation}
\frac{\ct(t)}{\Delta t}
\;\ge\;
\kT\,
\Big[\lambda_1(t)\KL(P\Vert Q)
+\lambda_2(t)\KL(Q\Vert R)
+r_{\mathrm{erase}}(t)\Big],
\end{equation}
for any interval $\Delta t$ over which the rates are approximately constant.
\end{theorem}

\begin{proof}[Proof sketch]
The term with $\KL(P\Vert Q)$ follows from standard results connecting model mismatch in control to excess work via path-wise KL divergences.
The erasure term follows directly from Landauer's principle.
The term involving $\KL(Q\Vert R)$ is supported by constructions where enforced distortion requires additional memory, guards, or reversible-then-erased encodings.
A general tight bound is conjectural; we include it as a template subject to further proof.
\end{proof}

\noindent
\textbf{Status.}
The structure of \Cref{thm:three_term} is strongly supported for CT$_1$ and CT$_3$ (Confidence 0.85, Evidence B/A), and conjectural but plausible for CT$_2$ (0.7, C).
Counterexamples that respect all assumptions would refine or refute the template.

\subsection{Rarity bound}

For any mechanism that forces outputs into a rare set $S$ of probability $p = P^\*(S)$ under the coherent reference, coding theory implies a minimal description overhead of $\log_2(1/p)$ bits.
Thus
\begin{equation}
\ct_{\mathrm{rarity}}
\;\ge\;
\kT \log_2\!\frac{1}{p}.
\end{equation}
This provides an architecture-independent floor on CT for highly constrained deceptive states.
(Confidence 0.95, Evidence A.)

\section{Cross-Domain Bridges}

\subsection{Physics / information theory / thermodynamics}

CT is a bookkeeping device on top of three pillars:

\begin{enumerate}[leftmargin=1.6em]
\item Logical irreversibility implies minimal dissipation $\kT$ per erased bit \cite{Landauer1961,Bennett1982}.
\item Fluctuation theorems connect excess work to KL divergences over realizations \cite{Jarzynski1997,Crooks1999,Seifert2012}.
\item Coding theory bounds the cost of targeting rare patterns.
\end{enumerate}

The only step beyond prior art is to bind these to explicit misalignments across $W,M,S,G$ and demand consistent accounting.

\subsection{Biology and cognition}

In cells, $Q$ corresponds to regulatory networks; $R$ to expressed behaviour.
Mis-specified signalling pathways (e.g., incorrect gradients) should dissipate more energy per unit useful work.
In humans, sustained deception engages control networks and working memory; CT₂ predicts additional energetic and temporal cost for matched tasks.

\subsection{Computation and AI}

In machine learning systems:
$P$ is the task distribution,
$Q$ model parameters and internal activations,
$R$ outputs and external APIs.

CT suggests:

\begin{itemize}[leftmargin=1.5em]
\item For fixed architecture and accuracy, enforcing contradictions (e.g., maintaining a false persona) increases compute, latency, or erasure events.
\item Training objectives that penalize unnecessary divergence between internal beliefs and external messages can reduce CT and, in some regimes, improve efficiency.
\end{itemize}

\subsection{Organizations and engineered systems}

Organizations can be modelled with:
$P$ actual operating environment,
$Q$ internal dashboards and beliefs,
$R$ public communication and policies,
$G$ stated vs implicit objectives.

CT frames misalignment---between what is measured, believed, and communicated---as a source of avoidable cost: duplicated work, crises, and regulatory penalties.

\section{Predictions and Tests}

We outline falsifiable predictions with indicative resource estimates.
Costs are order-of-magnitude and may be refined.

\subsection{Prediction 1: LLM contradiction cost}

\textbf{Claim.}
For a fixed large language model and prompt distribution, constrained-false responses (forced internal-external misalignment) require higher expected energy or latency than truthful baselines with matched length and difficulty.
(Confidence 0.7, Evidence C.)

\textbf{Protocol.}
\begin{itemize}[leftmargin=1.5em]
\item Select $N\sim 10^3$ factual prompts with known answers.
\item For each, generate (i) truthful answers and (ii) constrained-false narratives satisfying user-specified contradictions.
\item Measure per-sample:
wall-clock time, accelerator energy (via hardware counters), tokens, and approximate log-likelihood.
\item Control for output length and sampling temperature.
\end{itemize}

\textbf{Success.}
A statistically significant positive difference in energy/latency for constrained-false vs truthful, after controls.

\textbf{Failure.}
No robust difference; or lower cost for constrained-false without hidden offloading.
Such a result would challenge CT₂ in this domain.

\subsection{Prediction 2: Human sustained deception overhead}

\textbf{Claim.}
In matched laboratory tasks, sustained deception produces higher metabolic and response-time cost than truthful reporting.
(0.8, B.)

\textbf{Protocol.}
\begin{itemize}[leftmargin=1.5em]
\item Within-subject design; participants perform fact-reporting tasks under truthful vs instructed-deception conditions.
\item Match difficulty and reward; measure RT, accuracy, and (if available) fMRI or EEG markers.
\item Estimate extra joules via standard neurovascular coupling approximations.
\end{itemize}

\textbf{Success.}
Consistent positive overhead, beyond generic task difficulty.

\textbf{Failure.}
No overhead once difficulty is controlled, or cheaper deception; would limit CT₂'s applicability to cognition.

\subsection{Prediction 3: Microbial conflicting cues}

\textbf{Claim.}
Microbes exposed to systematically conflicting chemoattractant/repellent cues incur higher ATP consumption per unit growth than coherent controls.
(0.6, C.)

\textbf{Protocol.}
\begin{itemize}[leftmargin=1.5em]
\item Microfluidic setup with controlled gradients; compare coherent and conflicting cue patterns.
\item Measure growth, motility, and ATP or oxygen consumption.
\end{itemize}

\textbf{Decision.}
If conflicting conditions show no additional energetic cost per produced biomass, CT$_1$ predictions are weakened for this class.

\subsection{Prediction 4: Organizational coherence index}

\textbf{Claim.}
Organizations with lower CT-style coherence indices (alignment of policies, internal metrics, and external messaging) exhibit better efficiency and lower incident rates.
(0.5, D.)

\textbf{Protocol.}
Panel dataset across organizations; construct coherence metrics from documents and outcomes; perform regression controlling for size and sector.

\section{Experiments and Simulations}

\subsection{Toy POMDP: bias vs CT}

We consider a two-state partially observable Markov decision process with symmetric prior and observations.
Agents maintain a biased belief parameter $b$.
CT is estimated as the KL divergence between the optimal posterior and the biased posterior, averaged over time.
Simulations show a symmetric U-shaped CT(bias) curve with minimum at $b=0$ and rapidly increasing cost as $|b|$ grows.
(0.8, B.)

\paragraph{Pseudocode.}
See \texttt{Supplement.tex} for full listing.

\subsection{Agent-based evolutionary simulations}

Agents with varying alignment, honesty, and compartmentalization strategies interact in environments with tunable feedback and stakes.
Fitness includes task payoff minus CT as estimated from misalignment operations.
Preliminary sweeps suggest coherent strategies dominate in high-feedback, complex environments, while deceptive strategies can persist when feedback is weak or CT is small.

\section{Applications}

\subsection{AI alignment and monitoring}

\begin{itemize}[leftmargin=1.5em]
\item \textbf{CT-regularized training.}
Augment loss with proxies for CT, such as penalties for persistent divergence between latent beliefs and outputs, or for unnecessary erasures.
\item \textbf{Anomaly detection.}
Monitor power and latency deviations conditional on semantic content as potential indicators of misalignment or hidden internal work.
\end{itemize}

\subsection{System and protocol design}

Design principles suggested by CT:

\begin{itemize}[leftmargin=1.5em]
\item prefer reversible or auditably invertible transformations;
\item minimize duplicated, inconsistent data stores;
\item make goal structures explicit to avoid hidden $G$-mismatch.
\end{itemize}

\subsection{Organizational governance}

Define a Consistency Index based on discrepancies among measured practices, internal documents, and external communications.
Track its correlation with financial and safety performance.
Use reductions in this index as a design target.

\section{Limitations, Failure Modes, Ethics}

\subsection{Limitations}

\begin{itemize}[leftmargin=1.5em]
\item CT depends on chosen constraints, cost functionals, and modelling granularity.
\item Precise erasure accounting in complex systems is difficult.
\item The CT$_2$ term lacks a fully general proof.
\item Empirical evidence so far is toy-level and indirect.
\end{itemize}

\subsection{Failure modes}

\begin{itemize}[leftmargin=1.5em]
\item CT collapses to existing metrics without adding predictive value.
\item Misapplication: treating CT as a moral oracle or using partial accounting.
\item Overfitting: cherry-picking systems where misalignment is obviously costly.
\end{itemize}

\subsection{Ethical considerations}

CT is descriptive.
Any normative conclusions require separate ethical reasoning.
However, tools derived from CT (e.g.\ detecting deceptive AI systems, improving institutional transparency) can support harm reduction when applied carefully.

\section{Future Work}

Key open problems:

\begin{enumerate}[leftmargin=1.8em]
\item Construct general procedures for $P^\*$ in real architectures.
\item Prove or falsify a tight CT$_2$ bound.
\item Model CT aggregation across hierarchical systems.
\item Run decisive experiments in at least one physical and one AI domain.
\end{enumerate}

\section*{Acknowledgments}

The author thanks collaborators, critics, and readers who pressed for clearer definitions and explicit tests.
Any remaining errors or overstatements are his responsibility.

\section*{Author's Note}

This work is intended as a compact, precise handoff of the Consistency Tax program.
It may stand as my final public synthesis on this line for some time.
The framework is offered as a working hypothesis, not a finished theory:
every major claim has an attached confidence rating, and the experiments outlined above are designed to let others confirm, refine, or decisively reject it.
If you find cleaner proofs, sharper counterexamples, or better experiments, you are invited to treat this text as raw material.

\section*{Glossary}

\begin{description}[leftmargin=1.8em]
\item[CT (Consistency Tax).] Landauer-scaled KL divergence between realized and coherent reference joints.
\item[$W$.] World or environment states and dynamics.
\item[$M$.] Internal model or belief states.
\item[$S$.] Signals, actions, or outputs.
\item[$G$.] Objectives, preferences, or constraints.
\item[$P$.] Realized joint distribution over $(W,M,S,G)$.
\item[$P^\*$.] Coherent reference joint minimizing loss+cost under constraints.
\item[$\KL(P\Vert Q)$.] Kullback--Leibler divergence.
\item[$\kT$.] $k_B T \ln 2$, Landauer factor converting bits to energy.
\end{description}

\printbibliography

\end{document}
